\begin{figure}[h]
\includegraphics[width=0.95\textwidth]{figures/mem_model}      
\caption{\openshmem Memory Model}
\label{fig:mem_model}                                               
\end{figure}      
%
An \openshmem program consists of data objects that are private to each \ac{PE}
and data  objects that are remotely accessible by all \acp{PE}. Note that the latter 
of these two data objects are symmetric data objects. Private data
objects are stored in the local memory of each \ac{PE} and can only be accessed
by the \ac{PE} itself; these data objects cannot be accessed by other \acp{PE}
via \openshmem routines. Private data objects follow the memory model of
\Cstd or \Fortran. Symmetric data objects, however, can be accessed by
remote \acp{PE} using \openshmem routines. Each symmetric data object has a
corresponding object with the same name, type, and size on all PEs where that object is
accessible via the \openshmem \ac{API}\footnote{For efficiency reasons,
the same offset (from an arbitrary memory address) for symmetric data
objects might be used on all \acp{PE}. Further discussion about symmetric heap
layout and implementation efficiency can be found in section
\ref{subsec:shfree}}.  (For the definition of what is accessible, see the
descriptions for \FUNC{shmem\_pe\_accessible} and \FUNC{shmem\_addr\_accessible}
in sections \ref{subsec:shmem_pe_accessible} and
\ref{subsec:shmem_addr_accessible}.) Symmetric data objects accessed via typed
\openshmem interfaces are required to be naturally aligned based on their type
requirements and underlying architecture.  In \openshmem the following kinds of
data objects are symmetric:
%
\begin{itemize}
  \item \Fortran data objects in common blocks or with the  SAVE  attribute.
      These data objects must not be defined in a dynamic shared object (DSO).
  \item Global and static \Cstd and \Cpp variables. These data objects must
      not  be defined in a DSO.
  \item \Fortran arrays allocated with \textit{shpalloc} 
  \item \Cstd and \Cpp data allocated by \textit{shmem\_malloc}
\end{itemize}       

\openshmem dynamic memory allocation routines (\textit{shpalloc} and
\textit{shmem\_malloc}) allow collective allocation of symmetric data objects
on a special memory region called the \emph{Symmetric Heap}. The
symmetric heap is created, during the execution of a program, at an 
implementation-specific memory location. The symmetric heap may reside in different
memory regions on different \acp{PE}. Figure~\ref{fig:mem_model} shows how
\openshmem implements a \ac{PGAS} model using remotely accessible symmetric data
objects and private data objects when executing an \openshmem program.
Symmetric data objects are stored on the symmetric heap or in the global/static
memory section of each \ac{PE}. 
